\documentclass[OPS,lsstdraft,toc]{lsstdoc}

% lsstdoc documentation: https://lsst-texmf.lsst.io/lsstdoc.html

% Package imports go here.

% Local commands go here.

% To add a short-form title:
% \title[Short title]{Title}
\title{Release Scenarios for LSST Data}

% Optional subtitle
% \setDocSubtitle{A subtitle}

\author{%
William O'Mullane, Phil Marshall, Leanne Guy 
}

\setDocRef{LSO-011}

\date{\today}

% Optional: name of the document's curator
% \setDocCurator{The Curator of this Document}

\setDocAbstract{%
A first go at describing some release scenarios for LSST
}

% Change history defined here.
% Order: oldest first.
% Fields: VERSION, DATE, DESCRIPTION, OWNER NAME.
% See LPM-51 for version number policy.
\setDocChangeRecord{%
  \addtohist{1}{YYYY-MM-DD}{Unreleased.}{William O'Mullane, Phil Marchall, Leanne Guy }
}

\begin{document}

% Create the title page.
% Table of contents is added automatically with the "toc" class option.

\mkshorttitle
%switch to \maketitle if you wan the title page and toc


% ADD CONTENT HERE ... a file per section can be good for editing
\section{Introduction} \label{sec:intro}
This is in the file intro.tex. Put your text here.

The appendices for the \appref{sec:bib} and \appref{sec:acronyms} are defined in the main file LDM-000.tex.

The main bibliography file is in the lsst-texmf/texmf/bibtex/bib so you can refer to documents such as \citeds{LDM-294} (from lsst.bib)  or papers like \cite{2008arXiv0805.2366I} (from refs\_ads.bib). You may make a PR to add new refs to these files.

Acronyms like BOE and BAC will be picked up by generateAcronyms.py which is in lsst-texmf/bin -- that needs to be in the PATH.


\appendix
% Include all the relevant bib files.
% https://lsst-texmf.lsst.io/lsstdoc.html#bibliographies
\section{References} \label{sec:bib}
\bibliography{lsst,lsst-dm,refs_ads,refs,books}

%Make sure lsst-texmf/bin/generateAcronyms.py is in your path
\section{Acronyms used in this document}\label{sec:acronyms}
\addtocounter{table}{-1}
\begin{longtable}{|l|p{0.8\textwidth}|}\hline
\textbf{Acronym} & \textbf{Description}  \\\hline

LDM & LSST Data Management (document handle) \\\hline
LSO & LSST Science Operations (document handle) \\\hline
LSST & Large Synoptic Survey Telescope \\\hline
MREFC & NSF's Major Research Equipment and Facilities \\\hline
OPS & OPerationS \\\hline
US & United States \\\hline
\end{longtable}

\end{document}
